\documentclass[mcrownvopaper,10pt,twopage,onecolumn,final]{memoir}
\usepackage{graphicx}
\usepackage{gnos}
\usepackage[T1]{fontenc}
\usepackage{hyperref}

\setcounter{tocdepth}{1}
\setcounter{secnumdepth}{0}
%\settrimmedsize{8in}{5in}{*}
%
%\setlength{\trimtop}{\stockheight} % \trimtop = \stockheight
%\addtolength{\trimtop}{-\paperheight} % - \paperheight
%\setlength{\trimedge}{\stockwidth} % \trimedge = \stockwidth
%\addtolength{\trimedge}{-\paperwidth} % - \paperwidth

\setlrmarginsandblock{.3in}{*}{1}
\setulmarginsandblock{.4in}{.6in}{*}
\setheadfoot{\headheight}{0.1in}
\setheaderspaces{*}{0.07in}{*}
\checkandfixthelayout 

\setlength{\belowcaptionskip}{-10pt}

\begin{document}
\tableofcontents
\chapter{Reading}
The problems in this book are almost all reading problems. They are
not going to tax your judgement by asking you to find the largest point
on the board, choose the direction of play, or ponder the relative merits
of profit and outer strength. Instead, they are going to ask you to work
out sequences of moves that capture, cut, link up, make good shape, or
accomplish some other clear tactical objective.

A good player tries to read out such tactical problems in his head
before he puts the stones on the board. He looks before he leaps.
Frequently he does not leap at all; many of the sequences his reading
uncovers are stored away for future reference, and in the end never
carried out. This is especially true in a professional game, where the two
hundred or so moves played are only the visible part of an iceberg of
implied threats and possibilities, most of which stays submerged. You
may try to approach the game at that level, or you may, like most of us,
think your way from one move to the next as you play along, but in
either case it is your reading ability more than anything else that
determines your rank. 

There is an element of natural talent involved, but for the most part
reading ability is developed through study and experience. As you
become familiar with various positions and shapes you will find certain
moves, called tesuji, that come up again and again, and once you learn
them your reading will become much faster and more accurate. There
are also certain habits of thinking to be acquired, which this chapter will
try to illustrate.
\newpage

The first principle in reading is to start with a definite purpose. There
is no better way to waste time than to say to yourself, `I wonder what
happens if I play here', and start tracing out sequences aimlessly. Tactics
musts serve strategy. Start by asking yourself what you would like to
accomplish in the position in question, then start hunting for the
sequence that accomplishes it. Once you have your goal clearly in mind
the right move, if it exists, will be much easier to find.

With the goal set, reading is a matter of working your way through a
mental tree diagram of possible moves. You should be systematic and
thorough. Start with the obvious move, followed by the obvious
counter-move, the obvious counter-move to that, and so on until you
have a sequence that ends in success for one side and failure for the
other. Then take the last move made by the side that failed and try other
possibilities. If they all fail too, go back to the same side's move before
that and do the same thing again. It is important to work from the back
toward the front of the sequence, to avoid leaving things out. Eventually
you will arrive at a conclusion, and hopefully it will be correct.

As an example, let us take the question of whether Black can cut off the
five white stones in the lower portion of Dia.\ 1. Both players want to
know the answer to this question, but let us imagine ourselves as Black
and follow his thought processes as he reads the problem out.

\begin{figure}[ht]
\centering
    {\gnos%
    (((((>\\
    +!!++]\\
    ++@!!]\\
    ++@@!]\\
    ++@++]\\
    ++@!+]\\
    ++@!+]\\
    +++@!]\\
    +++@!]\\
    ++*@!]\\
    +++@@]\\
    +++++]\\
    }
    Dia.\ 1: Black to Play
\end{figure}

\newpage

Since he is trying to separate the two stones marked {\gnosfontsize{8}\gnos%
t}, the obvious
move to start with is 1 in Dia.\ 2. The obvious counter-move, White 2 in
Dia.\ 3, fails because of Black 3. Black 1 looks promising, but we must
consider other possible counter-moves by White.
\gnosfontsize{12}
\begin{figure}[ht]
\begin{minipage}[c]{0.4\linewidth}
\centering
    {\gnos%
    (((((>\\
    +!!++]\\
    ++@!!]\\
    ++@@t]\\
    ++@+{\gnosb\char1}]\\
    ++@t+]\\
    ++@!+]\\
    +++@!]\\
    +++@!]\\
    ++*@!]\\
    +++@@]\\
    +++++]\\
    }
    Dia.\ 2
\end{minipage}\hfill
\begin{minipage}[c]{0.4\linewidth}
    \centering    
    {\gnos%
    (((((>\\
    +!!++]\\
    ++@!!]\\
    ++@@!]\\
    ++@{\gnosw\char2}@]\\
    ++@!+]\\
    ++@!{\gnosb\char3}]\\
    +++@!]\\
    +++@!]\\
    ++*@!]\\
    +++@@]\\
    +++++]\\
    }
    Dia.\ 3
\end{minipage}
\end{figure}

The next most obvious counter-move is White 2 in Dia.\ 4, which
aims at going over the black stone at \textit{a} or under it at \textit{b}. For Black 3 we
start by blocking White's path as in Dia.\ 5 and letting him cut. Black
gives atari at 5, White connects at 6, and Black is dead. Are there any
better possibilities for Black 5? No, so this Black 3 fails.
Next comes Black 3 in Dia.\ 6. After White links underneath Black has
what looks like a tesuji at 5, but it comes to nothing. This Black 3 fails
too.

By now Black may be ready to conclude that White 2 works, but he
still has other Black 3's to try. Sooner or later the hane at 3 in Dia.\ 7 is
going to come to light. This is a real tesuji, the eye-stealing tesuji, and if
you know it you probably spotted it immediately. It stops White from
linking up, and White cannot cut at \textit{a} because of shortage of liberties,
(that is, he would be putting himself into atari). This is still true after
White 4 and Black 5; the white stones are cut off and dead.
\begin{figure}[ht]
    \begin{minipage}[c]{0.24\linewidth}
    \centering
        {\gnos%
        (((((>\\
        +!!++]\\
        ++@!!]\\
        ++@@!]\\
        ++@{\gnosEmptyLbl{\textit{a}}}@\gnosEmptyLbl{\textit{b}}\\
        ++@!{\gnosw\char2}]\\
        ++@!+]\\
        +++@!]\\
        +++@!]\\
        ++*@!]\\
        +++@@]\\
        +++++]\\
        }
        Dia.\ 4
    \end{minipage}
    \begin{minipage}[C]{0.24\linewidth}
        \centering    
        {\gnos%
        (((((>\\
        +!!++]\\
        ++@!!]\\
        ++@@!]\\
        ++@{\gnosw\char4}@{\gnosb\char3}\\
        ++@!!{\gnosb\char5}\\
        ++@!{\gnosw\char6}]\\
        +++@!]\\
        +++@!]\\
        ++*@!]\\
        +++@@]\\
        +++++]\\
        }
        Dia.\ 5
    \end{minipage}
    \begin{minipage}[c]{0.24\linewidth}
        \centering    
        {\gnos%
        (((((>\\
        +!!++]\\
        ++@!!]\\
        ++@@!]\\
        ++@{\gnosb\char3}@{\gnosw\char4}\\
        ++@!{\gnosw\char2}{\gnosw\char6}\\
        ++@!+{\gnosb\char5}\\
        +++@!]\\
        +++@!]\\
        ++*@!]\\
        +++@@]\\
        +++++]\\
        }
        Dia.\ 6
    \end{minipage}
    \begin{minipage}[c]{0.24\linewidth}
        \centering    
        {\gnos%
        (((((>\\
        +!!++]\\
        ++@!!]\\
        ++@@!]\\
        ++@\gnosEmptyLbl{\textit{a}}@{\gnosb\char5}\\
        ++@!!{\gnosb\char3}\\
        ++@!+{\gnosw\char4}\\
        +++@!]\\
        +++@!]\\
        ++*@!]\\
        +++@@]\\
        +++++]\\
        }
        Dia.\ 7
    \end{minipage}
\end{figure}

So the White 2 we have been investigating in Dias. 4 to 7 turns out to be
a failure; that only means that other, less obvious White 2's have to be
tested. The next candidate might be the hane shown in Dia.\ 8.
\newpage

If Black plays 3 in Dia.\ 9, White will connect at 4 and be threatening
to link up with either \textit{a} or \textit{b}. Black cannot defend against both of these
threats, so he has failed. In this kind of situation \textit{a} and \textit{b} are called miai;
if one player takes one of them, the other player can take the other.

\begin{figure}[ht]
    \begin{minipage}[c]{0.24\linewidth}
    \centering
        {\gnos%
        (((((>\\
        +!!++]\\
        ++@!!]\\
        ++@@!]\\
        ++@+{\gnosb\char1}{\gnosw\char2}\\
        ++@!+]\\
        ++@!+]\\
        +++@!]\\
        +++@!]\\
        ++*@!]\\
        +++@@]\\
        +++++]\\
        }
        Dia.\ 8
    \end{minipage}
    \begin{minipage}[C]{0.24\linewidth}
        \centering    
        {\gnos%
        (((((>\\
        +!!++]\\
        ++@!!]\\
        ++@@!]\\
        ++@\gnosEmptyLbl{\textit{a}}@!\\
        ++@!{\gnosb\char3}\gnosEmptyLbl{\textit{b}}\\
        ++@!{\gnosw\char4}]\\
        +++@!]\\
        +++@!]\\
        ++*@!]\\
        +++@@]\\
        +++++]\\
        }
        Dia.\ 9
    \end{minipage}
    \begin{minipage}[c]{0.24\linewidth}
        \centering    
        {\gnos%
        (((((>\\
        +!!++]\\
        ++@!!]\\
        ++@@!\gnosEmptyLbl{\textit{b}}\\
        ++@{\gnosw\char4}@{\gnosw\char2}\\
        ++@!\gnosEmptyLbl{\textit{a}}{\gnosb\char3}\\
        ++@!{\gnosb\char5}]\\
        +++@!]\\
        +++@!]\\
        ++*@!]\\
        +++@@]\\
        +++++]\\
        }
        Dia.\ 10
    \end{minipage}
    \begin{minipage}[c]{0.24\linewidth}
        \centering    
        {\gnos%
        (((((>\\
        +!!++]\\
        ++@!!]\\
        ++@@!\gnosEmptyLbl{\textit{c}}\\
        ++@+{\gnosb\char1}]\\
        ++@!+]\\
        ++@!\gnosEmptyLbl{\textit{b}}\gnosEmptyLbl{\textit{a}}\\
        +++@!]\\
        +++@!]\\
        ++*@!]\\
        +++@@]\\
        +++++]\\
        }
        Dia.\ 11
    \end{minipage}
\end{figure}
Black 3 in Dia.\ 9 failed, but Black 3 in Dia.\ 10 succeeds. If White
cuts at 4, Black has a snap-back at 5; if White plays 4 at \textit{a}, Black
captures at \textit{b}; and if White connects at \textit{b}, Black can play 4, 5, or \textit{a}. This
eliminates the hane for White 2.
\newpage

White's resources are fast disappearing, and we must now turn to
rather unlikely-looking choices, such as \textit{a}, \textit{b}, and even \textit{c} in Dia.\ 11, for
White 2. Each of these, however, can quickly be eliminated. See if you
can find answers to them for yourself; only White \textit{a} is at all tricky, (it
invites a mistake in which Black captures two of the white stones but
misses the rest).

If you have dealt with the moves in Dia.\ 11, then a total of six White
2's have been shown to fail. Does that mean that Black 1 is established?
Not yet, for there is one White 2 left, the least obvious and strongest
move of all.

The last arrow in White's quiver is the one-point jump to the edge in
Dia. 12. It guards the cutting point at \textit{a} and hence threatens to cut at \textit{b}. If
Black connects at 3 in Dia. 13, White can link up with 4, and Black 3 in
Dia. 14 runs into a move that we have seen before. These two Black 3's
are failures.

\begin{figure}[ht]
    \begin{minipage}[c]{0.24\linewidth}
    \centering
        {\gnos%
        (((((>\\
        +!!++]\\
        ++@!!]\\
        ++@@!]\\
        ++@\gnosEmptyLbl{\textit{b}}@]\\
        ++@!+{\gnosw\char2}\\
        ++@!\gnosEmptyLbl{\textit{a}}]\\
        +++@!]\\
        +++@!]\\
        ++*@!]\\
        +++@@]\\
        +++++]\\
        }
        Dia.\ 12
    \end{minipage}
    \begin{minipage}[C]{0.24\linewidth}
        \centering    
        {\gnos%
        (((((>\\
        +!!++]\\
        ++@!!]\\
        ++@@!]\\
        ++@{\gnosb\char3}@{\gnosw\char4}\\
        ++@!+!\\
        ++@!+]\\
        +++@!]\\
        +++@!]\\
        ++*@!]\\
        +++@@]\\
        +++++]\\
        }
        Dia.\ 13
    \end{minipage}
    \begin{minipage}[c]{0.24\linewidth}
        \centering    
        {\gnos%
        (((((>\\
        +!!++]\\
        ++@!!]\\
        ++@@!]\\
        ++@+@]\\
        ++@!{\gnosb\char3}!\\
        ++@!{\gnosw\char4}]\\
        +++@!]\\
        +++@!]\\
        ++*@!]\\
        +++@@]\\
        +++++]\\
        }
        Dia.\ 14
    \end{minipage}
    \begin{minipage}[c]{0.24\linewidth}
        \centering    
        {\gnos%
        (((((>\\
        +!!++]\\
        ++@!!]\\
        ++@@!]\\
        ++@+@{\gnosb\char3}\\
        ++@!+!\\
        ++@!+]\\
        +++@!]\\
        +++@!]\\
        ++*@!]\\
        +++@@]\\
        +++++]\\
        }
        Dia.\ 15
    \end{minipage}
\end{figure}

Boldness may succeed where caution fails, so next Black tries
blocking White's way directly with 3 in Dia.\ 15. At first, this seems to
work. White cannot cut at 4 in Dia.\ 16 because Black will cut him right
back with 5. Since 4 fails, there is no way White can get through to the
corner; he has put up a good fight, but it looks as if he has lost in the
end. Just to be on the safe side, however, Black had better doublecheck
for an alternative to White 4 in Dia.\ 16.
\newpage

And sure enough, there is White 4 in Dia.\ 17. Black connects at 5
and although White is cut off, he can live by playing 6.
There is something maddening to Black about reading to this point,
proving that no matter how White answers Black 1 he can be cut off,
only to discover that the cut-off group can live. Patiently Black goes on
and tests other Black 1's, like the one in Dia.\ 18, but they all fail.
The conclusion he comes to is that Dia.\ 19 is the best sequence
for both sides.

Since he has put so much thought into it, Black may be tempted to
play out Dia.\ 19 even though it is not a real success; at least it gives him
some profit in sente, and maybe White will miss the tesuji at 2.

\begin{figure}[ht]
    \begin{minipage}[c]{0.24\linewidth}
    \centering
        {\gnos%
        (((((>\\
        +!!++]\\
        ++@!!]\\
        ++@@!]\\
        ++@{\gnosw\char4}@@\\
        ++@!+]\\
        ++@!{\gnosb\char5}]\\
        +++@!]\\
        +++@!]\\
        ++*@!]\\
        +++@@]\\
        +++++]\\
        }
        Dia.\ 16
    \end{minipage}
    \begin{minipage}[C]{0.24\linewidth}
        \centering    
        {\gnos%
        (((((>\\
        +!!++]\\
        ++@!!]\\
        ++@@!]\\
        ++@{\gnosb\char5}@@\\
        ++@!{\gnosw\char4}!\\
        ++@!+{\gnosw\char6}\\
        +++@!]\\
        +++@!]\\
        ++*@!]\\
        +++@@]\\
        +++++]\\
        }
        Dia.\ 17
    \end{minipage}
    \begin{minipage}[c]{0.24\linewidth}
        \centering    
        {\gnos%
        (((((>\\
        +!!++]\\
        ++@!!]\\
        ++@@!]\\
        ++@+{\gnosb\char3}]\\
        ++@!{\gnosw\char2}]\\
        ++@!{\gnosb\char1}{\gnosw\char4}\\
        +++@!]\\
        +++@!]\\
        ++*@!]\\
        +++@@]\\
        +++++]\\
        }
        Dia.\ 18
    \end{minipage}
    \begin{minipage}[c]{0.24\linewidth}
        \centering    
        {\gnos%
        (((((>\\
        +!!++]\\
        ++@!!]\\
        ++@@!]\\
        ++@{\gnosb\char5\char1\char3}\\
        ++@!{\gnosw\char4\char2}\\
        ++@!+{\gnosw\char6}\\
        +++@!]\\
        +++@!]\\
        ++*@!]\\
        +++@@]\\
        +++++]\\
        }
        Dia.\ 19
    \end{minipage}
\end{figure}

There are two reasons, however, why Black should restrain himself.
The first is that moves like these should be saved for use as ko threats.
Most games involve at least one ko fight, and the player who squanders
his threats before the ko is going to be sorry. If Black leaves the position
alone White is not likely to bother making a defensive move, so the
opportunity to play 1 will still be there later on.

The second reason is that there is always the chance of having made
a reading mistake. Especially in a non-urgent position like this, you can
afford to turn your attention elsewhere, then come back later for a
second look. Re-examining positions that you have already read out is a
good way to spend the time waiting for your opponent to play; it often
turns up moves that were missed before.
\newpage

In the position we are considering, for example, if Black looks again
he may finally see 5 in Dia.\ 20, which destroys White's eye shape while
inflicting shortage of liberties on him to keep him from cutting at a.
Now he has the truth. He does not have to play 1 at once, but he knows
that when the time comes, the white stones are there for the taking.

\begin{figure}[ht]
    \centering
        {\gnos%
        (((((>\\
        +!!++]\\
        ++@!!]\\
        ++@@!]\\
        ++@\gnosEmptyLbl{\textit{a}}{\gnosb\char1\char3}\\
        ++@!{\gnosw\char4\char2}\\
        ++@!+{\gnosb\char5}\\
        +++@!]\\
        +++@!]\\
        ++*@!]\\
        +++@@]\\
        +++++]\\
        }
        Dia.\ 20
    \end{figure}

When you have a sequence that almost works, like the one in Dia.\ 19,
it is a good idea not to give up on it. Often changing just one move,
or changing the order of moves, or reading just one move further is all
that is needed.

What about the positions that are simply too hard to read out? As far
as possible, they should be left alone. Future developments may alter
them, and the unreadable may become readable, and anyway you lose
much more by having a lot of stones captured in a sequence that fails
than by letting your opponent defend where you could have destroyed
him. In the latter case, while your opponent is defending you get two
moves in a row elsewhere on the board. In the former case there is no
compensation. Sometimes, of course, you have to push ahead blindly,
but remember that it is weak players who are always playing in
situations they cannot read out, and strong players who refrain from
playing even when they have the situation completely read out.
\newpage

Most of the rest of this book consists of examples of tesuji and
problems on which you can practice your reading. One word of warning
about the answers to the problems is necessary. In general there will be
only one or two answer diagrams, showing how the correct answer
succeeds against the opponent's strongest resistance. For the problem
read out in this chapter, only the variations of diagrams 7, 20, and
perhaps 10 would appear in the answer diagrams. The rest of the reading
would be left up to you. Occasionally a wrong answer is shown as a
pitfall, and marked `failure'.

Since the opponent's strongest resistance to the correct answer fails,
it will not usually be the best move for him to make in actual play.
Faced with Black 1 in Dia.\ 20, for instance, White's best response is not
the `strongest' move at 2, but rather no move at all. In the endgame
White should play the hane, (2 at 3), and connect, a variation that would
not appear among the answer diagrams. If you respect your opponent's
reading ability you will want to avoid many of the even-numbered
moves in the answer diagrams of this book.

It took us twenty diagrams to get through one problem in this
chapter, but most of the problems coming up will not turn out to be so
complicated, and even the hard ones should not take so long once you
have gotten a grasp of tesuji. The importance of learning tesuji is that
you learn where to look for the answer, and can go straight to the move
that works without having to waste time thinking about moves that fail.

\chapter{Capture the Cutting Stones}
Diagram 1 shows the kind of move that this chapter is about. White
has one stone on the outside, partly surrounded by black stones but
ready to make a dash for the open. Black 1 traps it, blocking its escape
and capturing it.

Diagram 2 shows the same type of operation, except that now Black 1
captures two white stones. Try as they may, they cannot escape. In the
next few pages you will meet more advanced tesuji for trapping enemy
stones out in the open or for running them to earth at the edge of the
board.

\begin{figure}[ht]
    \begin{minipage}[c]{0.33\linewidth}
        \centering    
        {\gnos%
        (((((((>\\
        +++!!++]\\
        ++@@!+!]\\
        +++!@!+]\\
        ++{\gnosb\char1}+@@!]\\
        ++++++!]\\
        +++++++]\\
        }
        Dia.\ 1
    \end{minipage}%
    \begin{minipage}[c]{0.33\linewidth}
        \centering    
        {\gnos%
        ((((@((>\\
        +++@!!!]\\
        ++@@@!+!\\
        +++!!@!]\\
        ++{\gnosb\char1}++@@]\\
        +++++++]\\
        +++++++]\\
        }
        Dia.\ 2
    \end{minipage}%
    \begin{minipage}[c]{0.33\linewidth}
        \centering    
        {\gnos%
        (((((((!\\
        +++@!!!]\\
        ++@@@@!]\\
        +++!!@!]\\
        ++\gnosEmptyLbl{\textit{a}}++@@]\\
        +++++++]\\
        +++++++]\\
        }
        Dia.\ 3
    \end{minipage}
\end{figure}

What makes moves like this worth playing is not so much the size of
the capture—Black is getting only two points in Dia.\ 1 and four points
in Dia.\ 2—but the fact that the captured stones were cutting stones. If
White, instead of Black, played 1 in Dia.\ 1 for example, the black stones
would be split into two very weak groups, one or the other of which
would almost surely die.

Contrast this with Dia.\ 3, where the two white stones are not cutting
stones. Black could capture them with a, but that would be only a four-
point move of little significance. Black should ignore the enemy
stones, extend farther from his position, and try to surround a much
larger area.
\section{The Knight's Move Tesuji}
\begin{figure}[ht]
    \begin{minipage}[c]{0.5\linewidth}
        \centering    
        {\gnos%
        +++@+++\\
        !!!@+++\\
        !@@!T++\\
        +++!+++\\
        +@@+\gnosEmptyLbl{\textit{a}}++\\
        +++++++\\
        +++++++\\
        }
        Dia.\ 1
    \end{minipage}%
    \begin{minipage}[c]{0.5\linewidth}
        \centering    
        {\gnos%
        +++@+++\\
        !!!@{\gnosb\char5}++\\
        !@@!@{\gnosw\char4}+\\
        +++!{\gnosw\char2}{\gnosb\char3\char7}\\
        +@@{\gnosw\char8}{\gnosb\char1}{\gnosw\char6}+\\
        +++++++\\
        +++++++\\
        }
        Dia.\ 2
    \end{minipage}
\end{figure}

\noindent
\textbf{Dia.\ 1} Black wants to capture the two white stones in the center. Black
\textit{a}, the obvious move, does not work because White can push out
between \textit{a} and {\gnosfontsize{9}\gnos T} and escape with a series of ataris. 

\noindent
\textbf{Dia.\ 2} Ataris at 4 and 6 spring White free. Fortunately there is a play
that succeeds where Black 1 fails.

\gnosfontsize{12}
\begin{figure}[ht]
    \begin{minipage}[c]{0.5\linewidth}
        \centering    
        {\gnos%
        +++@+++\\
        !!!@+++\\
        !@@!T++\\
        +++!{\gnosw\char2}{\gnosb\char3}+\\
        +@@++{\gnosb\char1}+\\
        +++++++\\
        +++++++\\
        }
        Dia.\ 3
    \end{minipage}%
    \begin{minipage}[c]{0.5\linewidth}
        \centering    
        {\gnos%
        +++@+++\\
        !!!@+++\\
        !@@!@++\\
        +++!+++\\
        +@@+{\gnosw\char2}{\gnosb\char1}+\\
        ++++{\gnosb\char3}++\\
        +++++++\\
        }
        Dia.\ 4
    \end{minipage}
\end{figure}

\noindent
\textbf{Dia.\ 3} This Black 1 is the tesuji; observe its knight's-move relation to
Black {\gnosfontsize{9}\gnos T}, the weak stone that caused the trouble in the previous
diagram. After 2 and 3 an atari against Black {\gnosfontsize{9}\gnos T} would accomplish
nothing, and White is trapped much as in Dia.\ 2 on the previous page. 

\noindent
\textbf{Dia.\ 4} Nor can White escape this way. Black 3 stops him.

\begin{figure}[ht]
    \begin{minipage}[c]{0.5\linewidth}
        \centering    
        {\gnos%
        ++++++++]\\
        ++++++@+]\\
        +++++!!!]\\
        ++++@+!@]\\
        +++!@@!@]\\
        +!+!!!@@]\\
        ++@!@@!!]\\
        +++@++@@]\\
        )))))))).\\
        }
        Problem 1. White to play and capture the cutting stones.
    \end{minipage}%
    \begin{minipage}[c]{0.5\linewidth}
        \centering    
        {\gnos%
        ++++++++]\\
        +@++++!+]\\
        +++++!++]\\
        ++++@@!!]\\
        +++!@!!@@\\
        ++!@!@@@]\\
        +@+@!+@+@\\
        +++@!!!@@\\
        )))))))!.\\
        }
        Problem 2. White to play and capture the cutting stones.
    \end{minipage}
\end{figure}

\begin{figure}[ht]
    \begin{minipage}[c]{0.5\linewidth}
        \centering    
        {\gnos%
        ++++++++]\\
        ++++++@+]\\
        ++{\gnosw\char1}++!!!]\\
        ++++@+!@]\\
        +++t@@!@]\\
        +!+!!!@@]\\
        ++@!@@!!]\\
        +++@++@@]\\
        )))))))).\\
        }
        Ans.\ to Prob.\ 1
    \end{minipage}%
    \begin{minipage}[c]{0.5\linewidth}
        \centering    
        {\gnos%
        ++++++++]\\
        +++{\gnosw\char1}++@+]\\
        ++{\gnosb\char2}{\gnosw\char3}+!!!]\\
        +++{\gnosb\char4}@+!@]\\
        +++!@@!@]\\
        +!+!!!@@]\\
        ++@!@@!!]\\
        +++@++@@]\\
        )))))))).\\
        }
        Dia.\ 1a
    \end{minipage}
\end{figure}
\noindent
\textbf{Answer to Problem 1.} White 1, a knight's move away from the weak
stone {\gnosfontsize{9}\gnos t}, does the job.
\begin{figure}[!ht]
    \begin{minipage}[c]{0.5\linewidth}
        \centering    
        {\gnos%
        ++++++++]\\
        +@++++!+]\\
        ++{\gnosw\char1}++!++]\\
        +++{\gnosb\char2}@@!!]\\
        ++{\gnosw\char3}!@!!@@\\
        ++!@!@@@]\\
        +@+@!+@+@\\
        +++@!!!@@\\
        )))))))!.\\
        }
        Ans.\ to Prob.\ 2
    \end{minipage}%
    \begin{minipage}[c]{0.5\linewidth}
        \centering    
        {\gnos%
        ++++++++]\\
        +@+{\gnosw\char5}++!+]\\
        ++{\gnosw\char1}{\gnosb\char2}{\gnosb\char4}!++]\\
        +++{\gnosw\char3}@@!!]\\
        +++!@!!@@\\
        ++!@!@@@]\\
        +@+@!+@+@\\
        +++@!!!@@\\
        )))))))!.\\
        }
        Dia.\ 2a
    \end{minipage}%
\end{figure}

\noindent
\textbf{Answer to Problem 2.} White 1 traps the black stones.

\noindent
\textbf{Dia.\ 1a.} This is the wrong knight's move. Black 2 makes a neat escape.

\noindent
\textbf{Dia. 2a.} If Black plays 2, White has a short ladder.

\section{The Loose Ladder Tesuji}
\noindent
\textbf{Dia.\ 1.} If Black is going to get any kind of result out of this position he
has to capture the pair of white stones to the right of \textit{a}, but how? An
atari at a would not work.

\noindent
\textbf{Dia.\ 2.} Black 1 is the tesuji; it sets up a loose ladder.

\gnosfontsize{12}
\begin{figure}[ht]
    \begin{minipage}[c]{0.33\linewidth}
        \centering    
        {\gnos%
        (((((((>\\
        +++@@@!]\\
        ++\gnosEmptyLbl{\textit{a}}!!@!]\\
        ++++@!!]\\
        ++++@!+]\\
        ++!+!@@]\\
        ++++!++]\\
        +++++@+]\\
        +++++++]\\
        }
        Dia.\ 1
    \end{minipage} 
    \begin{minipage}[c]{0.33\linewidth}
        \centering    
        {\gnos%
        (((((((>\\
        +++@@@!]\\
        +++!!@!]\\
        ++{\gnosb\char1}+@!!]\\
        ++++@!+]\\
        ++!+!@@]\\
        ++++!++]\\
        +++++@+]\\
        +++++++]\\
        }
        Dia.\ 2
    \end{minipage} 
    \begin{minipage}[c]{0.32\linewidth}
        \centering    
        {\gnos%
        (((((((>\\
        +{\gnosb\char5}{\gnosw\char4}@@@!]\\
        +{\gnosb\char3}{\gnosw\char2}!!@!]\\
        ++@\gnosEmptyLbl{\textit{a}}@!!]\\
        +++\gnosEmptyLbl{\textit{b}}@!+]\\
        ++!+!@@]\\
        ++++!++]\\
        +++++@+]\\
        +++++++]\\
        }
        Dia.\ 3
    \end{minipage}%
\end{figure}

\noindent
\textbf{Dia.\ 3.} Black guides White firmly to the edge of the board with 3 and 5,
not only trapping the fleeing stones but capturing the whole corner. If at
any point White plays \textit{a}, Black \textit{b} puts him in atari and hastens his end.

\begin{figure}[ht]
    \begin{minipage}[c]{0.5\linewidth}
        \centering    
        {\gnos%
        \char91+++++++++\\
        \char91+++++++++\\
        \char91+++++++++\\
        \char91++++++@++\\
        \char91+!!@+++++\\
        \char91!+@!!+@+*\\
        \char91!@@@!++++\\
        @@+!+@@+++\\
        ,)))))))))\\
        }
        Problem\ 1. White to play and capture the cutting stones
    \end{minipage}% 
    \begin{minipage}[c]{0.5\linewidth}
        \centering    
        {\gnos%
        \char91+++++++++++\\
        \char91+++++++++++\\
        \char91+!!!+++++++\\
        \char91++@++++++++\\
        \char91+++++++++++\\
        \char91+@@@@@++*!+\\
        \char91!!+!@!!++++\\
        \char91+++!!@@++++\\
        ,)))))))))))\\
        }
        Problem\ 2 Black to play and capture White's cutting stones
    \end{minipage}
\end{figure}

\begin{figure}[ht]
    \begin{minipage}[c]{0.5\linewidth}
        \centering    
        {\gnos%
        \char91+++++++++\\
        \char91{\gnosw\char11\char9}+++++++\\
        \char91{\gnosb\char10\char8}{\gnosw\char5\char3}+++++\\
        \char91{\gnosw\char7}{\gnosb\char6\char4\char2}{\gnosw\char1}+@++\\
        \char91+!!@+++++\\
        \char91!+@!!+@+*\\
        \char91!@@@!++++\\
        @@+!+@@+++\\
        ,)))))))))\\
        }
        Ans.\ to Prob.\ 1
    \end{minipage}% 
    \begin{minipage}[c]{0.5\linewidth}
        \centering    
        {\gnos%
        \char91+++++++++++\\
        \char91+++++++++++\\
        \char91+!!!+++++++\\
        \char91++@{\gnosb\char11\char9\char7\char5}++++\\
        \char91+++{\gnosw\char10\char8\char6\char4}{\gnosb\char3}+++\\
        \char91+@@@@@{\gnosw\char2}+*!+\\
        \char91!!+!@!!{\gnosb\char1}+++\\
        \char91+++!!@@++++\\
        ,)))))))))))\\
        }
        Ans.\ to Prob.\ 2
    \end{minipage}
\end{figure}
\noindent
\textbf{Answer to problem 1.} White 1 and 7 are the key plays.

\noindent
\textbf{Answer to problem 2.} This time the sequence starts with an atari.

\section{The Slapping Tesuji}
\noindent
\textbf{Dia.\ 1.} Black is trying to bring his two stones out into the open with {\gnosfontsize{9}\gnos T},
(although he is doing it wrong, as will quickly become clear). Can
White stop him? The series of ataris, i.e. the ladder, that starts with
White \textit{a} is broken by Black {\gnosfontsize{9}\gnos T}, so White must look for something else.

\noindent
\textbf{Dia.\ 2.} In this shape White 1 is the tesuji. It makes White \textit{a} a real threat,
so if Black is going to resist he must either connect at a himself or try to
slip out with \textit{b}.

\noindent
\textbf{Dia.\ 3.} But if Black connects at 2, White has him in a loose ladder with
3 and the rest.

\gnosfontsize{12}
\begin{figure}[ht]
    \begin{minipage}[c]{0.33\linewidth}
        \centering    
        {\gnos%
        \char91+++++++\\
        \char91++!++++\\
        \char91+!@++++\\
        \char91+!@\gnosEmptyLbl{\textit{a}}T++\\
        \char91!@!!+++\\
        \char91@@@!+++\\
        \char91++@!+!+\\
        \char91++@@!++\\
        ,)))))))\\
        }
        Dia.\ 1
    \end{minipage}% 
    \begin{minipage}[c]{0.33\linewidth}
        \centering    
        {\gnos%
        \char91+++++++\\
        \char91++!\gnosEmptyLbl{\textit{b}}+++\\
        \char91+!@+{\gnosw\char1}++\\
        \char91+!@\gnosEmptyLbl{\textit{a}}@++\\
        \char91!@!!+++\\
        \char91@@@!+++\\
        \char91++@!+!+\\
        \char91++@@!++\\
        ,)))))))\\
        }
        Dia.\ 2
    \end{minipage}% 
    \begin{minipage}[c]{0.33\linewidth}
        \centering    
        {\gnos%
        \char91+++++++\\
        \char91++!++++\\
        \char91+!@+!++\\
        \char91+!@{\gnosb\char2}@{\gnosw\char3}+\\
        \char91!@!!{\gnosb\char4}{\gnosw\char5}+\\
        \char91@@@!{\gnosb\char6}++\\
        \char91++@!{\gnosw\char7}!+\\
        \char91++@@!++\\
        ,)))))))\\
        }
        Dia.\ 3
    \end{minipage}
\end{figure}

\noindent
\textbf{Dia.\ 4.} Black 6 here, an attempt to set up a snap-back, bows before
White 7.

\noindent
\textbf{Dia.\ 5.} What about the other possible Black 2? White 3 gives atari, and
from there on the moves are the same as before, except that the loose
ladder becomes an ordinary ladder.

\noindent
\textbf{Dia.\ 6.} To return to Black's original move, if he wants to escape he has
to make an empty triangle with 1. Empty triangles are bad shape, but at
least he has a chance to split White up and attack.

\begin{figure}[ht]
    \begin{minipage}[c]{0.33\linewidth}
        \centering    
        {\gnos%
        \char91+++++++\\
        \char91++!++++\\
        \char91+!@+!++\\
        \char91+!@@@!+\\
        \char91!@!!@!+\\
        \char91@@@!+{\gnosb\char6}{\gnosw\char7}\\
        \char91++@!+!+\\
        \char91++@@!++\\
        ,)))))))\\
        }
        Dia.\ 4
    \end{minipage}%
    \begin{minipage}[c]{0.33\linewidth}
        \centering    
        {\gnos%
        \char91+++++++\\
        \char91++!{\gnosb\char2}+++\\
        \char91+!@{\gnosw\char3}!++\\
        \char91+!@{\gnosb\char4}@{\gnosw\char5}+\\
        \char91!@!!+++\\
        \char91@@@!+++\\
        \char91++@!+!+\\
        \char91++@@!++\\
        ,)))))))\\
        }
        Dia.\ 5
    \end{minipage}
    \begin{minipage}[c]{0.33\linewidth}
        \centering    
        {\gnos%
        \char91+++++++\\
        \char91++!++++\\
        \char91+!@{\gnosb\char1}+++\\
        \char91+!@++++\\
        \char91!@!!+++\\
        \char91@@@!+++\\
        \char91++@!+!+\\
        \char91++@@!++\\
        ,)))))))\\
        }
        Dia.\ 6
    \end{minipage}
\end{figure}

\noindent
\textbf{Problem 1.} Black to play and capture the cutting stones.

\noindent
\textbf{Problem 2.} Black to play and capture the cutting stones. Be sure you
have read out the whole sequence correctly.

\begin{figure}[ht]
    \begin{minipage}[c]{0.57\linewidth}
        \centering    
        {\gnos%
        \char91++++++++++++\\
        \char91+!!!+!+!!+++\\
        \char91++@+++++@+++\\
        \char91++++++!+++++\\
        \char91+!+++++@@@++\\
        \char91++!+@!!@!+!!\\
        \char91!!@@+@@!!+++\\
        \char91!@++++++++++\\
        ,))))))))))))\\
        }
        Problem 1
    \end{minipage}%
    \begin{minipage}[c]{0.43\linewidth}
        \centering    
        {\gnos%
        \char91++++++++++\\
        \char91++++++++++\\
        \char91++++++++++\\
        \char91+++++!++++\\
        \char91+@@@@+++++\\
        \char91!!*!@!!@++\\
        \char91+++!!@@+++\\
        \char91++!@@+++++\\
        ,))))))))))\\
        }
        Problem 2
    \end{minipage}
\end{figure}
\newpage

\begin{figure}[ht]
    \begin{minipage}[c]{0.57\linewidth}
        \centering    
        {\gnos%
        \char91++++++++++++\\
        \char91+!!!+!+!!+++\\
        \char91++@+++{\gnosb\char3\char5}@+++\\
        \char91+++++{\gnosb\char1}!{\gnosw\char4}++++\\
        \char91+!++++{\gnosw\char2}@@@++\\
        \char91++!+@!!@!+!!\\
        \char91!!@@+@@!!+++\\
        \char91!@++++++++++\\
        ,))))))))))))\\
        }
        Ans.\ to Prob.\ 1
    \end{minipage}%
    \begin{minipage}[c]{0.43\linewidth}
        \centering    
        {\gnos%
        \char91++++++++++\\
        \char91++++++++++\\
        \char91++{\gnosb\char7}+{\gnosb\char5\char3}++++\\
        \char91+++{\gnosw\char6\char4}!{\gnosb\char1}+++\\
        \char91+@@@@{\gnosw\char2}++++\\
        \char91!!*!@!!@++\\
        \char91+++!!@@+++\\
        \char91++!@@+++++\\
        ,))))))))))\\
        }
        Ans.\ to Prob.\ 2
    \end{minipage}
\end{figure}

\noindent
\textbf{Answer to problem 1.} Black 1 is the tesuji, rnd the rest is simple.

\noindent
\textbf{Answer to problem 2.} Here the important point, aside from the tesuji at
1, is seeing when to jump ahead. Any move other than 7 would fail.

\section{The Clamping Tesuji}
\textbf{Dia.\ 1.} There may seem to be no way for White to capture anything in
this position. If he cuts at \textit{a}, for example, Black gets away with \textit{b}.

\noindent
\textbf{Dia.\ 2.} There is a tesuji, however: the clamping move at 1. Its effect is
to make miai of \textit{a} and \textit{b}.

\noindent
\textbf{Dia.\ 3.} For Black to connect at 2 is pointless. White 3 leaves him with
no room to wriggle.
\begin{figure}[ht]
    \begin{minipage}[c]{0.33\linewidth}
        \centering    
        {\gnos%
        (((((((>\\
        +++++++]\\
        ++@++++]\\
        ++++!!+]\\
        +++\gnosEmptyLbl{\textit{b}}@!+]\\
        ++@\gnosEmptyLbl{\textit{a}}@@!]\\
        ++!+@!!]\\
        +++!!@+]\\
        +++++@+]\\
        ++++@++]\\
        +++++++]\\
        }
        Dia.\ 1
    \end{minipage}%
    \begin{minipage}[c]{0.34\linewidth}
        \centering    
        {\gnos%
        (((((((>\\
        +++++++]\\
        ++@++++]\\
        ++++!!+]\\
        ++{\gnosw\char1}+@!+]\\
        +\gnosEmptyLbl{\textit{b}}@\gnosEmptyLbl{\textit{a}}@@!]\\
        ++!+@!!]\\
        +++!!@+]\\
        +++++@+]\\
        ++++@++]\\
        +++++++]\\
        }
        Dia.\ 2
    \end{minipage}%
    \begin{minipage}[c]{0.33\linewidth}
        \centering    
        {\gnos%
        (((((((>\\
        +++++++]\\
        ++@++++]\\
        ++++!!+]\\
        ++!+@!+]\\
        +{\gnosw\char3}@{\gnosb\char2}@@!]\\
        ++!+@!!]\\
        +++!!@+]\\
        +++++@+]\\
        ++++@++]\\
        +++++++]\\
        }
        Dia.\ 3
    \end{minipage}%
\end{figure}

\noindent
\textbf{Dia.\ 4.} But if he extends out with 2, White 3 severs him. Black's 4 does
not work because White 5 captures his stones in a snap-back. Similarly,
if Black played 2 at 4, White would answer at 5, leaving 2 and 3 as miai.
Clamping tesuji do not always involve snap-back, but they are sometimes
hard to see, so we have given you three problems this time.

\begin{figure}[ht]
    \begin{minipage}[c]{0.50\linewidth}
        \centering    
        {\gnos%
        (((((((>\\
        +++++++]\\
        ++@++++]\\
        +++{\gnosw\char5}!!+]\\
        ++!{\gnosb\char4}@!+]\\
        +{\gnosb\char2}@{\gnosw\char3}@@!]\\
        ++!+@!!]\\
        +++!!@+]\\
        +++++@+]\\
        ++++@++]\\
        +++++++]\\
        }
        Dia.\ 4
    \end{minipage}%
    \begin{minipage}[c]{0.50\linewidth}
        \centering    
        {\gnos%
        (((((((!((>\\
        +++++@!+!!]\\
        +@++++@+!@]\\
        +*++@+@@!@]\\
        +@+@!!@!@@]\\
        +++@++!!++]\\
        ++++++++@+]\\
        +++++++@++]\\
        ++++++++++]\\
        }
        Problem 1
    \end{minipage}%
\end{figure}

\noindent
\textbf{Problem 1.} White to play and capture the cutting stones.

\noindent
\textbf{Problem 2.} Black to play and halt White's escape. Don't be confused by
extraneous stones.

\noindent
\textbf{Problem 3.} White to play and capture the cutting stones.

\begin{figure}[ht]
    \begin{minipage}[c]{0.50\linewidth}
        \centering    
        {\gnos%
        (((((((((>\\
        ++++@++!@]\\
        +!+!@!!@+]\\
        \char42++@@+*@+]\\
        ++++!+!@+]\\
        +++++++++]\\
        +++++++++]\\
        +++++++++]\\
        }
        Problem 2
    \end{minipage}%
    \begin{minipage}[c]{0.50\linewidth}
        \centering    
        {\gnos%
        (((((((>\\
        +@@+++@]\\
        ++!@+@!]\\
        ++!+*+!]\\
        ++!@@!+]\\
        ++@!!++]\\
        +++++++]\\
        +++++++]\\
        }
        Problem 3
    \end{minipage}%
\end{figure}  
\newpage

\begin{figure}[ht]
    \begin{minipage}[c]{0.50\linewidth}
        \centering    
        {\gnos%
        (((((((!((>\\
        ++++{\gnosw\char3}@!+!!]\\
        +@++{\gnosw\char1}{\gnosb\char2}@+!@]\\
        +*+\gnosEmptyLbl{\textit{a}}@\gnosEmptyLbl{\textit{b}}@@!@]\\
        +@+@!!@!@@]\\
        +++@++!!++]\\
        ++++++++@+]\\
        +++++++@++]\\
        ++++++++++]\\
        }
        Ans.\ to Prob.\ 1
    \end{minipage}%
    \begin{minipage}[c]{0.50\linewidth}
        \centering    
                {\gnos%
        (((((((((>\\
        ++++@++!@]\\
        +!+!@!!@+]\\
        \char42++@@+\gnosEmptyLbl{\textit{a}}@+]\\
        ++++!{\gnosb\char1}!@+]\\
        ++++++\gnosEmptyLbl{\textit{b}}++]\\
        +++++++++]\\
        +++++++++]\\
        }
        Ans.\ to Prob.\ 2
    \end{minipage}%
\end{figure}  
\noindent
\textbf{Answer to problem 1.} White 1 is the clamping tesuji; it makes a and b
miai. If Black plays 2, White plays 3, and vice versa.

\noindent
\textbf{Answer to problem 2.} This time Black has to make his clamping move
right in between two white stones, but it still works. Again a and b are
miai, and the three white stones above a are cut off and done for.

\begin{figure}[ht]
    \begin{minipage}[c]{0.50\linewidth}
        \centering    
        {\gnos%
        (((((((>\\
        +@@+{\gnosb\char2}+@]\\
        ++!@{\gnosw\char1}@!]\\
        ++!+{\gnosw\char3}+!]\\
        ++!@@!+]\\
        ++@!!++]\\
        +++++++]\\
        +++++++]\\
        }
        Ans.\ to Prob.\ 3
    \end{minipage}%
    \begin{minipage}[c]{0.50\linewidth}
        \centering    
        {\gnos%
        (((((((>\\
        +@@\gnosEmptyLbl{\textit{a}}{\gnosw\char3}\gnosEmptyLbl{\textit{c}}@]\\
        ++!@{\gnosw\char1}@!]\\
        ++!\gnosEmptyLbl{\textit{b}}{\gnosb\char2}\gnosEmptyLbl{\textit{d}}!]\\
        ++!@@!+]\\
        ++@!!++]\\
        +++++++]\\
        +++++++]\\
        }
        Dia.\ 3a
    \end{minipage}
\end{figure}

\noindent
\textbf{Answer to problem 3.} White 1 is a surprising tesuji. Black's best
response is 2, but White 3 gives atari and if Black connects he will still
be in atari.

\noindent
\textbf{Dia.\ 3a.} Black's play in this diagram only magnifies his loss. After
White 3, a and b are miai on one side and c and d are miai on the other,
and the three stones including Black 2 are captured.

\section{The Nose Tesuji}

\textbf{Dia.\ 1.} Black is in a position to capture the upper pair of white stones,
but he must be careful because the danger of White a is staring him in
the face. A non-contact play would be too slow.

\noindent
\textbf{Dia.\ 2.} Black 1 hits White squarely on the nose, so to speak. More to the
point, it defends against a and catches the white stones in a loose ladder.

\noindent
\textbf{Dia.\ 3.} Black 3 and 5 drive White to the edge of the board and allow
him no escape.

\begin{figure}[ht]
    \begin{minipage}[c]{0.33\linewidth}
        \centering    
        {\gnos%
        (((((((>\\
        +++++++]\\
        ++++@@+]\\
        +++!!@+]\\
        ++++@!+]\\
        ++@\gnosEmptyLbl{\textit{a}}@!+]\\
        ++++@!+]\\
        +++!!++]\\
        +++++++]\\
        }
        Dia.\ 1
    \end{minipage}%
    \begin{minipage}[c]{0.33\linewidth}
        \centering    
        {\gnos%
        (((((((>\\
        +++++++]\\
        ++++@@+]\\
        ++{\gnosb\char1}!!@+]\\
        ++++@!+]\\
        ++@\gnosEmptyLbl{\textit{a}}@!+]\\
        ++++@!+]\\
        +++!!++]\\
        +++++++]\\
        }
        Dia.\ 2
    \end{minipage}%
    \begin{minipage}[c]{0.33\linewidth}
        \centering    
        {\gnos%
        (((((((>\\
        ++{\gnosb\char5}{\gnosw\char4}+++]\\
        ++{\gnosb\char3}{\gnosw\char2}@@+]\\
        ++@!!@+]\\
        ++++@!+]\\
        ++@+@!+]\\
        ++++@!+]\\
        +++!!++]\\
        +++++++]\\
        }
        Dia.\ 3
    \end{minipage}%
\end{figure}


\noindent
\textbf{Problem 1.} White to play and capture the cutting stones. As usual, be
sure to read out the whole sequence.

\noindent
\textbf{Problem 2.} Black to play and capture the stone marked {\gnosfontsize{8}\gnos t}. The nose
tesuji is not the first move, but comes later.
\begin{figure}[ht]
    \begin{minipage}[c]{0.45\linewidth}
        \centering    
        {\gnos%
        \char91++++++++++\\
        \char91@@++++++++\\
        \char91!@++++++++\\
        \char91!!+@++++++\\
        \char91++!@!!!!@+\\
        \char91++!@!@+@@+\\
        \char91++!!@@++++\\
        ,))))))))))\\
        }
        Problem 1
    \end{minipage}%
    \begin{minipage}[c]{0.55\linewidth}
        \centering    
        {\gnos%
        \char91++++++++++++\\
        \char91++++++++!+++\\
        \char91@!!+!+++!@++\\
        \char91@!@++++@@!++\\
        \char91@!@!@++@!!++\\
        \char91@@!!+++t@+!+\\
        \char91@!+++++++@!+\\
        ,))))))))))))\\
        }
        Problem 2
    \end{minipage}%
\end{figure}
\section{The Gross-Girt Tesuji}

\noindent
\textbf{Dia.\ 1.} Black seems to be separated into two groups, a corner one and
an outside one, but there is a way for him to link them up and capture
the two white stones that stand in between.
It would be a fatal mistake for Black to start by giving atari at a.
Common sense might tell you that; Black can give atari either at a or
from the other side, so he should hold both ataris in reserve and wait
until one of them becomes effective.

\noindent
\textbf{Dia.\ 2.} The tesuji is the contact play at 1. White's best response is 2, and
Black cross-cuts with 3.

\noindent
\textbf{Dia. 3.} If White gives atari at 4, Black gives a counter-atari at 5 and
White cannot connect at 7 because of shortage of liberties. All he can do
is to capture at 6, letting Black have two stones with 7 or a.

\gnosfontsize{12}
\begin{figure}[ht]
    \begin{minipage}[c]{0.33\linewidth}
        \centering    
        {\gnos%
        (((((((>\\
        +++++++]\\
        +!!@@@+]\\
        !+@!!++]\\
        ++@@\gnosEmptyLbl{\textit{a}}!+]\\
        +++++++]\\
        +++++++]\\
        +++++!+]\\
        +++++++]\\
        }
        Dia.\ 1
    \end{minipage}% 
    \begin{minipage}[c]{0.33\linewidth}
        \centering    
        {\gnos%
        (((((((>\\
        +++++++]\\
        +!!@@@+]\\
        !+@!!++]\\
        ++@@+!{\gnosb\char3}]\\
        +++++{\gnosb\char1}{\gnosw\char2}]\\
        +++++++]\\
        +++++!+]\\
        +++++++]\\
        }
        Dia.\ 2
    \end{minipage}% 
    \begin{minipage}[c]{0.33\linewidth}
        \centering    
        {\gnos%
        (((((((>\\
        +++++++]\\
        +!!@@@+]\\
        !+@!!{\gnosb\char7}\gnosEmptyLbl{\textit{a}}]\\
        ++@@{\gnosb\char5}!@]\\
        ++++{\gnosw\char4}@!]\\
        +++++{\gnosw\char6}+]\\
        +++++!+]\\
        +++++++]\\
        }
        Dia.\ 3
    \end{minipage}% 
\end{figure}

\noindent
\textbf{Dia.\ 4.} If White gives atari from the other direction with 4, Black alters
5 accordingly.

\noindent
\textbf{Dia.\ 5.} There is one move to watch out for in this shape. Occasionally
when Black plays 1 White can resist with 2, threatening a and b.

\noindent
\textbf{Dia.\ 6.} But in the present position Black can foil White and make a big
capture with 3 etc.

\begin{figure}[ht]
    \begin{minipage}[c]{0.33\linewidth}
        \centering    
        {\gnos%
        (((((((>\\
        +++++++]\\
        +!!@@@+]\\
        !+@!!{\gnosb\char5}{\gnosw\char4}]\\
        ++@@{\gnosb\char7}!@{\gnosw\char6}\\
        +++++@!]\\
        +++++++]\\
        +++++!+]\\
        +++++++]\\
        }
        Dia.\ 4
    \end{minipage}% 
    \begin{minipage}[c]{0.33\linewidth}
        \centering    
        {\gnos%
        (((((((>\\
        +++++++]\\
        +!!@@@\gnosEmptyLbl{\textit{a}}]\\
        !+@!!+{\gnosw\char2}]\\
        ++@@+!+]\\
        +++++{\gnosb\char1}\gnosEmptyLbl{\textit{b}}]\\
        +++++++]\\
        +++++!+]\\
        +++++++]\\
        }
        Dia.\ 5
    \end{minipage}% 
    \begin{minipage}[c]{0.33\linewidth}
        \centering    
        {\gnos%
        (((((((>\\
        +++++++]\\
        +!!@@@{\gnosw\char4}]\\
        !+@!!{\gnosb\char5}!]\\
        ++@@{\gnosw\char6}!{\gnosb\char7}]\\
        +++{\gnosw\char10\char8}@{\gnosb\char3}]\\
        ++{\gnosb\char11}+{\gnosb\char9}++]\\
        +++++!+]\\
        +++++++]\\
        }
        Dia.\ 6
    \end{minipage}% 
\end{figure}
\newpage

\noindent
\textbf{Problem 1.} White to play and capture the cutting stone.

\noindent
\textbf{Problem 2.} Black to play and capture the cutting stone.

\begin{figure}[ht]
    \begin{minipage}[c]{0.5\linewidth}
        \centering    
        {\gnos% 
        \char91+@++++++\\
        \char91+++@++++\\
        \char91+++@!+++\\
        \char91+@++!+++\\
        \char91+@!!++++\\
        \char91+!@+++++\\
        \char91+!+@++@+\\
        \char91++++++++\\
        ,))))))))\\
        }
        Problem 1
    \end{minipage}%
    \begin{minipage}[c]{0.5\linewidth}
        \centering    
        {\gnos% 
        \char91+!!@!+++\\
        \char91+++!++++\\
        \char91++++++++\\
        \char91+!++++++\\
        \char91+++!@@++\\
        \char91+@@@!+++\\
        \char91+@!!!+!+\\
        \char91+@@!++++\\
        ,))))))))\\
        }
        Problem 2
    \end{minipage}

\end{figure}

\section{More Problems}
The following eleven problems range from the very easy, such as
numbers 3 and 4, to the moderately difficult, such as number 11. In each
of them the idea is to capture the cutting stones. In one of the problems,
(number 1), Black can save his cutting stones if he plays a certain way,
but White can get a good result anyhow.

The answers appear briefly on the following two pages. As usual, the
answer diagrams show moves that the player who loses the sequence
should leave unplayed.

\begin{figure}[ht]
    \begin{minipage}[c]{0.5\linewidth}
        \centering    
        {\gnos% 
        +++++++++++]\\
        +++++++++++]\\
        ++++!++++++]\\
        ++@@!++++++]\\
        +++!@!++@++]\\
        ++*!@+++*++]\\
        +@+!+@++@++]\\
        +++++++++++]\\
        ))))))))))).\\
        }
        1. White to play
    \end{minipage}%
    \begin{minipage}[c]{0.5\linewidth}
        \centering    
        {\gnos% 
        +++++++++++]\\
        +++++++++++]\\
        +++++++++++]\\
        +++++++++!+]\\
        ++++!++++++]\\
        ++@@!@@+@++]\\
        +!!@@!+++++]\\
        +++!!++++++]\\
        ))))))))))).\\
        }
        2. Black to play
    \end{minipage}

    \begin{minipage}[c]{0.32\linewidth}
        \centering    
        {\gnos% 
        +++++++]\\
        +++++++]\\
        +++++!+]\\
        ++++@!@]\\
        +++@!@@]\\
        ++!!!!@]\\
        +++@@@+]\\
        ))))))).\\
        }
        3. White to play
    \end{minipage}%
    \begin{minipage}[c]{0.31\linewidth}
        \centering    
        {\gnos% 
        ++++++]\\
        ++++@+]\\
        ++++!@]\\
        ++!+!@]\\
        ++@*@!]\\
        ++++@!]\\
        +++@!+!\\
        )))))!.\\
        }
        4. Black to play
    \end{minipage}%
    \begin{minipage}[c]{0.36\linewidth}
        \centering    
        {\gnos% 
        ++++++++]\\
        +++++@++]\\
        ++++++!!]\\
        +++++@!@@\\
        ++++!!@+]\\
        ++++!+@@@\\
        +++++!!@]\\
        ))))))@).\\
        }
        5. White to play
    \end{minipage}
    \begin{minipage}[c]{0.5\linewidth}
        \centering    
        {\gnos% 
        ++++++++]\\
        ++++++++]\\
        +++++*++]\\
        ++++++@+]\\
        ++++++++]\\
        ++++++++]\\
        ++++!+++]\\
        +++@@!++]\\
        ++!+@!++]\\
        +!+@+!++]\\
        ++++++++]\\
        )))))))).\\
        }
        6. White to play
    \end{minipage}%
    \begin{minipage}[c]{0.5\linewidth}
        \centering    
        {\gnos% 
        +++++++]\\
        +++++!+]\\
        +++@*++]\\
        +++++@!]\\
        +++@@!!]\\
        +++!!@+]\\
        +++++@+]\\
        ++++@++]\\
        ++++*!+]\\
        ++++!++]\\
        +++++++]\\
        ))))))).\\
        }
        7. Black to play
    \end{minipage}
\end{figure}

\begin{figure}[ht]
\end{figure}
\begin{figure}[ht]
\end{figure}
\begin{figure}[ht]
    \begin{minipage}[c]{0.36\linewidth}
        \centering    
        {\gnos% 
        ++++++++]\\
        ++++++++]\\
        ++++++++]\\
        ++++++++]\\
        ++++++++]\\
        ++++++++]\\
        ++++++++]\\
        ++++++++]\\
        ++++++++]\\
        ++++++++]\\
        ++++++++]\\
        )))))))).\\
        }
        8. White to play
    \end{minipage}%
\end{figure}

\chapter{Amputate the Cutting Stones}
The theme of this chapter is the same as that of the last: the capture
of small groups of enemy stones. The difference is that whereas before
the idea was to capture them by blocking their escape route, the idea
now is to capture them by detaching them from a larger body of enemy
stones, and the techniques differ accordingly. Usually the target stones
will be cutting stones, but we shall not be finicky about going after non-
cutting stones on occasion for the sheer profit of capturing them.
\section{Snap-back}
Snap-backs are the first really interesting tactics of the game that
most players learn; perhaps you can remember when you first
encountered one. For those who may not be sure of the term, the next
three diagrams present a quick review.

\section{The Throw-in Tesuji}
\section{The Squeeze Tesuji}
\section{Ladder-Building}
\section{The Placement Tesuji}
\section{More Problems}

\chapter{Ko}

\chapter{When Liberties Count}


\end{document}
